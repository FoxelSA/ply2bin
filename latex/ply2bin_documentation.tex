\documentclass[a4paper, 11pt]{article}

%\usepackage{../preambule}
\usepackage[latin1]{inputenc}
\usepackage[french]{babel}
\usepackage{amsmath}
%\usepackage{bbm}
\usepackage{lscape}
\usepackage{inputenc}
%\usepackage[refpage,french,noprefix]{nomencl}
\usepackage{lmodern}
\usepackage{url}
\usepackage{float, graphicx}
\usepackage[sectionbib]{chapterbib}

\author{ St\'ephane Flotron }
\title{\textbf{ply2bin documentation}}

\renewcommand{\tilde}{\widetilde}
\renewcommand{\t}{\mathbf{t}}

\begin{document}
   \maketitle
   
   \section*{Program description}
   
   The ply2bin software is a program which convert a point cloud in PLY file format into 
   a binary file readable for the DAV interface. 
   The keypoint of this sofware is to transform the (aligned) point cloud into the WebGL
   interface associated to the panorama.
   
   \section*{Mathematics}
   
   Suppose that we have an aligned point cloud in CH1903+ referential, and that we want to transform the points into
   the WebGL interface. In order to simplify the reading, let us introduce some notations.
   
   \begin{tabular}{cc}
        
   \end{tabular}
   
   \begin{equation}
       \begin{aligned}
           & X_{wgl} & = & \frac{1}{s} R_x(\pi) R_r R_2^T R_a^T X_{CH1903} + \frac{1}{s} R_x(\pi) R_r R_2^T [C_a -S] \\
            &&&       -R_x(\pi) R_r R_2^T \t_2 - R_x(\pi) R_r C_r.
       \end{aligned}
   \end{equation} 
   
   Let us define $\overline X $ as the 3D you need for projection onto panorama, and $\tilde X$ the 3D point you need 
   for measurments. Then 
   \begin{equation}
        \begin{aligned}
        \overline X & = && R_x(\pi) X_{wgl} \\
        \tilde X & = && s X_{wgl}
        \end{aligned}
   \end{equation}
   which is equivalent to 
      \begin{equation}
        \begin{aligned}
        \overline X & = && \frac{1}{s} R_r R_2^T R_a^T X_{CH1903} + \frac{1}{s} R_r R_2^T [C_a -S] -R_r R_2^T \t_2 - R_r C_r. \\
        \tilde X & = && R_x(\pi) R_r R_2^T R_a^T X_{CH1903} + R_x(\pi) R_r R_2^T [C_a -S] \\
            &&&       -s R_x(\pi) R_r R_2^T \t_2 - s R_x(\pi) R_r C_r.
        \end{aligned}
   \end{equation}



\end{document}